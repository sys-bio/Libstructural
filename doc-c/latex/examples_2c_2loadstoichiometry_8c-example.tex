\hypertarget{examples_2c_2loadstoichiometry_8c-example}{}\section{examples/c/loadstoichiometry.\+c}
This is an example of how to load a (unlabeled) stoichiometry matrix and read test details.


\begin{DoxyCodeInclude}
\textcolor{preprocessor}{#include <stdio.h>}           \textcolor{comment}{// for printf}
\textcolor{preprocessor}{#include <stdlib.h>}          \textcolor{comment}{// for malloc}
\textcolor{preprocessor}{#include <string.h>}          \textcolor{comment}{// for memset}
\textcolor{preprocessor}{#include <\hyperlink{libstructural_8h}{libstructural.h}>}   \textcolor{comment}{// the structural analysis library}

\textcolor{comment}{// construct simple stoichiometry matrix}
\textcolor{keywordtype}{void} GetMatrixFromSomeWhere(\textcolor{keywordtype}{double}** *oMatrix, \textcolor{keywordtype}{int} *nRows, \textcolor{keywordtype}{int} *nCols)
\{
        \textcolor{keywordtype}{int} numCols, numRows, i;
        numRows = 4; numCols = 3;
        
        \textcolor{comment}{// initialize memory needed}
        *oMatrix = (\textcolor{keywordtype}{double}**)malloc(\textcolor{keyword}{sizeof}(\textcolor{keywordtype}{double}*)*numRows); 
        memset(*oMatrix, 0, \textcolor{keyword}{sizeof}(\textcolor{keywordtype}{double}*)*numRows);
        
        \textcolor{keywordflow}{for} (i = 0; i < numRows; i ++)
        \{
                (*oMatrix)[i] = (\textcolor{keywordtype}{double}*)malloc(\textcolor{keyword}{sizeof}(\textcolor{keywordtype}{double})*numCols); 
                memset((*oMatrix)[i], 0, \textcolor{keyword}{sizeof}(\textcolor{keywordtype}{double})*numCols);
        \}
        
        \textcolor{comment}{// set non zero entries of the stoichiometry matrix}
        (*oMatrix)[0][1] = -1.0;    (*oMatrix)[0][2] =  1.0;    \textcolor{comment}{// ES}
        (*oMatrix)[1][0] =  1.0;    (*oMatrix)[1][2] = -1.0;    \textcolor{comment}{// S2}
        (*oMatrix)[2][0] = -1.0;    (*oMatrix)[2][1] =  1.0;    \textcolor{comment}{// S1}
        (*oMatrix)[3][1] =  1.0;    (*oMatrix)[3][2] = -1.0;    \textcolor{comment}{// E}
        
        
        \textcolor{comment}{// be sure to return number of rows and columns}
        *nRows = numRows;
        *nCols = numCols;
\}

\textcolor{keywordtype}{int} main (\textcolor{keywordtype}{int} argc, \textcolor{keywordtype}{char}** argv)
\{
        \textcolor{keywordtype}{int}      i;
        \textcolor{keywordtype}{int}      nRows;
        \textcolor{keywordtype}{int}      nCols;
        \textcolor{keywordtype}{double}** oMatrix;
        \textcolor{keywordtype}{char}*    sMessage;
        \textcolor{keywordtype}{int}      nLength;
        
        
        \textcolor{comment}{// get matrix to analyze from another part of the code}
        GetMatrixFromSomeWhere(&oMatrix, &nRows, &nCols);
        
        \textcolor{comment}{// load it into the structural analysis library}
        \hyperlink{libstructural_8cpp_aedb61ed1eaedaa02687d35f5b39d4736}{LibStructural\_loadStoichiometryMatrix} (oMatrix, nRows, nCols);
        
        \textcolor{comment}{// analyze the stoichiometry matrix using the QR method}
        \hyperlink{libstructural_8cpp_a505eacbf357b087e0d10a09feb0d69e2}{LibStructural\_analyzeWithQR}( &sMessage, &nLength);
        
        \textcolor{comment}{// print model overview}
        printf(\textcolor{stringliteral}{"%s"}, sMessage);
        
        \textcolor{comment}{// free the memory used by the message}
        \hyperlink{libstructural_8cpp_a43ec36f000082b338e1aad9cff3b0b23}{LibStructural\_freeVector}(sMessage);
        
        \textcolor{comment}{// obtain and print the test results}
        \hyperlink{libstructural_8cpp_a76fa7cdfb5333db9e78bf36ac6bc292e}{LibStructural\_getTestDetails}( &sMessage, &nLength );
        printf(\textcolor{stringliteral}{"%s"}, sMessage);
        
        \textcolor{comment}{// finally free the memory used by the message}
        \hyperlink{libstructural_8cpp_a43ec36f000082b338e1aad9cff3b0b23}{LibStructural\_freeVector}(sMessage);
        
        \textcolor{comment}{// and free the memory used to hold the stoichiometry matrix}
        \textcolor{keywordflow}{for} (i = 0; i < nRows; i++)
                free(oMatrix[i]);
        free(oMatrix);
        
        \textcolor{keywordflow}{return} 0;
        
\}

\textcolor{comment}{//The program above returns the following output: }
\textcolor{comment}{//-----------------------------------------------------------------------------}
\textcolor{comment}{//-----------------------------------------------------------------------------}
\textcolor{comment}{//STRUCTURAL ANALYSIS MODULE : Results }
\textcolor{comment}{//-----------------------------------------------------------------------------}
\textcolor{comment}{//-----------------------------------------------------------------------------}
\textcolor{comment}{//Size of Stochiometric Matrix: 4 x 3 (Rank is  2)}
\textcolor{comment}{//Nonzero entries in Stochiometric Matrix: 8  (66.6667% full)}
\textcolor{comment}{//}
\textcolor{comment}{//Independent Species (2) :}
\textcolor{comment}{//0, 1}
\textcolor{comment}{//}
\textcolor{comment}{//Dependent Species (2) :}
\textcolor{comment}{//2, 3}
\textcolor{comment}{//}
\textcolor{comment}{//L0 : There are 2 dependencies. L0 is a 2x2 matrix.}
\textcolor{comment}{//}
\textcolor{comment}{//Conserved Entities}
\textcolor{comment}{//1:  + 0 + 1 + 2}
\textcolor{comment}{//2:  + 0 + 3}
\textcolor{comment}{//-----------------------------------------------------------------------------}
\textcolor{comment}{//-----------------------------------------------------------------------------}
\textcolor{comment}{//Developed by the Computational Systems Biology Group at Keck Graduate Institute }
\textcolor{comment}{//and the Saurolab at the Bioengineering Departmant at  University of Washington.}
\textcolor{comment}{//Contact : Frank T. Bergmann (fbergman@u.washington.edu) or Herbert M. Sauro.   }
\textcolor{comment}{//}
\textcolor{comment}{//(previous authors) Ravishankar Rao Vallabhajosyula                   }
\textcolor{comment}{//-----------------------------------------------------------------------------}
\textcolor{comment}{//-----------------------------------------------------------------------------}
\textcolor{comment}{//}
\textcolor{comment}{//Testing Validity of Conservation Laws.}
\textcolor{comment}{//}
\textcolor{comment}{//Passed Test 1 : Gamma*N = 0 (Zero matrix)}
\textcolor{comment}{//Passed Test 2 : Rank(N) using SVD (2) is same as m0 (2)}
\textcolor{comment}{//Passed Test 3 : Rank(NR) using SVD (2) is same as m0 (2)}
\textcolor{comment}{//Passed Test 4 : Rank(NR) using QR (2) is same as m0 (2)}
\textcolor{comment}{//Passed Test 5 : L0 obtained with QR matches Q21*inv(Q11)}
\textcolor{comment}{//Passed Test 6 : N*K = 0 (Zero matrix)}
\end{DoxyCodeInclude}
 